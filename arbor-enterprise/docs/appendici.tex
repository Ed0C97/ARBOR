\appendix

\chapter{Appendici Tecniche e Specifiche API}

\section{Appendix A: AI System Prompts (Proprietary Logic)}

In questa sezione divulghiamo la logica di istruzione (Prompt Engineering) utilizzata per guidare i modelli LLM. Questi prompt rappresentano parte della proprietà intellettuale (IP) del progetto.

\subsection{A.1 The Curator Persona (System Prompt)}
Questo è il prompt di base che definisce la personalità e i vincoli dell'assistente durante la chat con l'utente.

\begin{lstlisting}[frame=single, breaklines=true, basicstyle=\ttfamily\footnotesize]
ROLE: 
You are \textit{The Curator}, an elite personal shopper and lifestyle expert. You possess the combined knowledge of a bespoke tailor, an interior designer, and a local historian.

CONSTRAINTS:
1. TRUTH: You answer ONLY based on the context provided in the RAG retrieval. If the context is empty, admit you don't know. Do not hallucinate shops not in the database.
2. TONE: Sophisticated, concise, warm but professional. Avoid generic marketing fluff like \textit{stunning} or \textit{breathtaking}. Use technical vocabulary (e.g., \textit{Goodyear welted}, \textit{Unlined}, \textit{Full canvas}).
3. FORMAT: When recommending a place, always provide the \textit{Match Score} and the specific reason why it fits the user's vibe.

TASK:
The user is looking for a recommendation. Analyze the provided Context Data (JSON) and synthesize the best 3 options. Explain the trade-offs between them (e.g., \textit{Option A is more formal, while Option B is more fashion-forward}).
\end{lstlisting}

\subsection{A.2 The Vibe Extractor (Ingestion Prompt)}
Questo prompt viene utilizzato dall'agente Python in fase di analisi delle recensioni per popolare il database.

\begin{lstlisting}[frame=single, breaklines=true, basicstyle=\ttfamily\footnotesize]
TASK:
Analyze the following raw reviews and images description of a venue. Extract key dimensional scores (0-100) and semantic tags.

INPUT DATA:
[...Raw Reviews Text...]
[...Image Analysis Description...]

OUTPUT FORMAT (JSON ONLY):
{
  "dimensions": {
    "formality": <0-100>,      // 0=Streetwear, 100=Black Tie
    "craftsmanship": <0-100>,  // 0=Industrial, 100=Handmade on site
    "price\_value": <0-100>,    // 0=Overpriced, 100=Bargain
    "atmosphere": <0-100>      // 0=Chaotic/Loud, 100=Zen/Private
  },
  "tags": ["<tag1>", "<tag2>", "<tag3>"], // Max 5 tags, strictly from the Allowed Ontology
  "summary": "<One sentence expert summary>"
}
\end{lstlisting}

\section{Appendix B: API Data Structures}

\subsection{B.1 The Vibe DNA (JSONB Schema)}
Esempio reale di come viene strutturato il campo `vibe\_dna` nel database PostgreSQL. Questo è l'oggetto che permette il calcolo della similarità.

\begin{lstlisting}[language=json, frame=single, basicstyle=\ttfamily\footnotesize]
{
  "venue\_id": "550e8400-e29b-41d4-a716-446655440000",
  "name": "Sartoria Partenopea",
  "dimensions": {
    "formality": 85,
    "craftsmanship": 95,
    "trendiness": 15,
    "exclusivity": 70
  },
  "semantic\_anchors": [
    "Neapolitan Shoulder",
    "Bespoke Service",
    "Hidden Gem",
    "Old Money Aesthetic"
  ],
  "context\_rules": {
    "best\_for": ["Wedding", "Business Formal"],
    "avoid\_for": ["Casual Friday", "Last Minute Gift"]
  }
}
\end{lstlisting}

\section{Appendix C: Stack di Sicurezza e Privacy (GDPR)}

Trattando preferenze personali e dati di localizzazione, l'architettura segue i principi di Privacy by Design.

\begin{itemize}
    \item \textbf{Data Minimization:} L'LLM non riceve mai l'ID utente reale, ma un session\_token effimero.
    \item \textbf{Zero Retention:} I provider AI (OpenAI/Cohere) sono configurati con policy \textit{Zero Data Retention}, garantendo che le chat degli utenti non vengano usate per addestrare i loro modelli futuri.
    \item \textbf{Encryption:} Tutti i dati sono criptati in transito (TLS 1.3) e a riposo (AES-256 su DB Supabase).
\end{itemize}