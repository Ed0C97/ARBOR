\chapter{The A.R.B.O.R. Architecture: God Mode}

\section{Overview: The Cognitive Stack}

A.R.B.O.R. non è una semplice applicazione web, ma una \textbf{Cognitive Architecture} distribuita che supera lo stato dell'arte attuale. Abbandoniamo l'approccio monolitico per adottare un sistema a \textbf{Sciame di Agenti (Agentic Swarm)} supportato da una \textbf{Trinità di Dati}.

Il sistema è progettato per gestire due flussi temporali:
\begin{itemize}
    \item \textbf{Sincrono (Real-Time):} Risposta all'utente in $<2$ secondi tramite Cache e Vettori.
    \item \textbf{Asincrono (Background):} Ingestione e ragionamento profondo tramite Grafo e LLM.
\end{itemize}

% Inizia la pagina orizzontale
\begin{landscape}
    \begin{figure}[p] % [p] forza la figura su una pagina dedicata senza testo
        \centering
        % Impostiamo larghezza e altezza massime mantenendo le proporzioni
        \includegraphics[width=\linewidth, height=1\textheight, keepaspectratio]{grafico_arbor.pdf}
        \caption{Master Blueprint: Architettura GraphRAG + Agentic Swarm}
        \label{fig:blueprint}
    \end{figure}
\end{landscape}
% Finisce la pagina orizzontale e torna verticale

\section{Layer 1: The Knowledge Trinity (La Memoria)}

Per permettere un ragionamento \textit{umano}, il sistema utilizza tre tipologie di database simultaneamente.

\paragraph{1. PostgreSQL (The Source of Truth)}
Contiene i fatti oggettivi e immutabili. Garantisce l'integrità transazionale (ACID).
\begin{itemize}
    \item \textbf{Dati:} Anagrafiche, Indirizzi, Prezzi, Orari.
    \item \textbf{Tecnologia:} PostgreSQL 16 con estensione PostGIS per la geolocalizzazione di precisione.
\end{itemize}

\paragraph{2. Qdrant (The Intuition)}
Il motore vettoriale scritto in Rust. Gestisce la \textit{ricerca per sensazione} (Vibe).
\begin{itemize}
    \item \textbf{Dati:} Embeddings (vettori a 1536 dimensioni) delle descrizioni e delle immagini.
    \item \textbf{Funzione:} Permette query come \textit{Trova un posto con atmosfera simile a questo}, impossibile per i database classici.
\end{itemize}

\paragraph{3. Neo4j (The Logic)}
Il Knowledge Graph. Mappa le relazioni invisibili e storiche.
\begin{itemize}
    \item \textbf{Dati:} Nodi (Persone, Brand, Stili) e Archi (Relazioni: \texttt{TRAINED\_BY}, \texttt{INSPIRED\_BY}).
    \item \textbf{Funzione:} Abilita il ragionamento transitivo: \textit{Consigliami questo sarto perché il suo maestro ha lavorato per Marinella}.
\end{itemize}

\section{Layer 2: The Agentic Swarm (Il Ragionamento)}

L'orchestrazione è gestita da \textbf{LangGraph}, che coordina agenti specializzati:

\begin{enumerate}
    \item \textbf{Intent Router:} Classifica la richiesta dell'utente e attiva l'agente giusto.
    \item \textbf{Vector Agent:} Interroga Qdrant per similarità estetica.
    \item \textbf{Historian Agent:} Interroga Neo4j per connessioni storiche.
    \item \textbf{Metadata Agent:} Interroga Postgres per filtri rigidi (prezzo, apertura).
    \item \textbf{The Curator:} Sintetizza i risultati e genera la risposta finale con il tono di voce del brand.
\end{enumerate}

\section{Layer 3: Ingestion \& Human-in-the-Loop}

La qualità del dato è garantita da una pipeline ibrida AI-Umana.
\begin{enumerate}
    \item \textbf{Scraper \& Vision AI:} Scaricano dati e analizzano foto per generare bozze.
    \item \textbf{Curator Dashboard:} Un'interfaccia dove gli esperti umani validano i dati prima della pubblicazione. Nessun dato entra nel sistema senza il marchio \textit{Vetted}.
\end{enumerate}

\section{Stack Tecnologico: Hybrid Performance}

Adottiamo una strategia \textit{Brain \& Muscle}:
\begin{itemize}
    \item \textbf{Python (Brain):} Usato per la logica AI (FastAPI, LangGraph) per la massima flessibilità.
    \item \textbf{Rust (Muscle):} Usato per i motori di calcolo pesanti (Qdrant, Pydantic Core) per la massima velocità e risparmio costi cloud.
\end{itemize}