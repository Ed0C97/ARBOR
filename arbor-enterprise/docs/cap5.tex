\chapter{Roadmap Operativa: From Zero to God Mode}

\section{Strategia di Esecuzione: The 12-Month Sprint}

La realizzazione di \textbf{Project A.R.B.O.R.} segue una metodologia Agile modificata per il Deep Tech. Non rilasceremo un prodotto incompleto, ma costruiremo strati di solidità progressiva.

La roadmap è divisa in 4 fasi trimestrali (Q1-Q4).

\section{Q1: The Foundation (Mesi 1-3)}
Obiettivo: Costruire l'infrastruttura dati e la pipeline di ingestione. Alla fine di questa fase, il sistema deve \textit{sapere} le cose, anche se non sa ancora \textit{parlare}.

\subsection{Mese 1: Infrastructure Setup}
\begin{itemize}
    \item \textbf{Cloud Environment:} Setup dei container Docker su Railway/AWS.
    \item \textbf{Database Trinity:} Inizializzazione delle istanze:
    \begin{itemize}
        \item \textbf{PostgreSQL:} Migrazioni schema SQL (Venues, Locations).
        \item \textbf{Qdrant:} Configurazione cluster Rust e indici HNSW.
        \item \textbf{Neo4j:} Definizione nodi e relazioni del Grafo.
    \end{itemize}
    \item \textbf{Repo Setup:} Configurazione CI/CD (GitHub Actions) e ambiente Python/Poetry.
\end{itemize}

\subsection{Mese 2: The Ingestion Engine (ETL)}
\begin{itemize}
    \item \textbf{Scraper Development:} Sviluppo dei bot Python per estrarre dati da Google Maps e Web.
    \item \textbf{Vision AI Integration:} Implementazione di GPT-4o Vision per analizzare le foto dei negozi e generare i primi \textit{Vibe Scores} automatici.
    \item \textbf{Curator Dashboard (v0.1):} Rilascio interno del pannello di controllo (Retool) per permettere al team editoriale di validare i dati scaricati.
\end{itemize}

\subsection{Mese 3: Data Population (Milano Pilot)}
\begin{itemize}
    \item \textbf{Massive Ingestion:} Caricamento dei primi 500 negozi su Milano (Focus: Sartoria, Artigianato, Hospitality).
    \item \textbf{Embedding Tuning:} Test dei vettori su Qdrant per assicurarsi che \textit{Elegante} e \textit{Formale} siano matematicamente vicini.
    \item \textbf{Graph Linking:} Creazione manuale/assistita delle prime relazioni storiche su Neo4j (e.g. \textit{Maestro X ha formato Sarto Y}).
\end{itemize}

\section{Q2: The Brain \& Logic (Mesi 4-6)}
Obiettivo: Sviluppare l'intelligenza. Alla fine di questa fase, il sistema sa ragionare e rispondere via API.

\subsection{Mese 4: The Agentic Swarm}
\begin{itemize}
    \item \textbf{LangGraph Setup:} Implementazione dell'orchestratore a grafo.
    \item \textbf{Agent Development:}
    \begin{itemize}
        \item Vector Agent: Connessione a Qdrant.
        \item Metadata Agent: Connessione a Postgres (SQL Tools).
        \item Historian Agent: Connessione a Neo4j (Cypher Tools).
    \end{itemize}
\end{itemize}

\subsection{Mese 5: The Curator Persona}
\begin{itemize}
    \item \textbf{System Prompt Engineering:} Raffinamento del tono di voce \textit{Porfirio}.
    \item \textbf{Intent Router:} Addestramento del classificatore per capire le intenzioni dell'utente.
    \item \textbf{Redis Caching:} Implementazione della Semantic Cache per ridurre i costi API e latenza.
\end{itemize}

\subsection{Mese 6: Internal Alpha (Dogfooding)}
\begin{itemize}
    \item \textbf{API Release (v1.0):} Rilascio degli endpoint FastAPI stabili.
    \item \textbf{Stress Test:} Simulazione di 100 utenti concorrenti per testare la tenuta di Qdrant e Postgre.g.
    \item \textbf{Quality Audit:} Il team di Porfirio Magazine testa le risposte per verificare l'accuratezza stilistica.
\end{itemize}

\section{Q3: The Experience (Mesi 7-9)}
Obiettivo: Dare un volto al sistema. Sviluppo delle interfacce utente Web e Mobile.

\subsection{Mese 7: Web App Integration}
\begin{itemize}
    \item \textbf{Next.js Development:} Sviluppo dell'interfaccia Chat e integrazione nel sito esistente di Porfirio Magazine.
    \item \textbf{Mapbox Integration:} Visualizzazione dei risultati su mappa interattiva customizzata (colori scuri/brandizzati).
    \item \textbf{Auth Integration:} Collegamento con Supabase Auth per gestire gli accessi.
\end{itemize}

\subsection{Mese 8: Mobile App (Flutter)}
\begin{itemize}
    \item \textbf{Core Development:} Porting delle funzionalità chat su iOS/Android.
    \item \textbf{Geolocation Features:} Implementazione della funzione \textit{Near Me} e notifiche push geolocalizzate.
    \item \textbf{App Store Submission:} Preparazione burocratica per Apple e Google Store.
\end{itemize}

\subsection{Mese 9: Closed Beta (\textit{The Vetted Club})}
\begin{itemize}
    \item \textbf{Soft Launch:} Invito a 500 utenti selezionati (HNWI / Amici del Brand).
    \item \textbf{Feedback Loop:} Raccolta bug e ottimizzazione UX.
    \item \textbf{Re-Ranking Tuning:} Ottimizzazione dell'algoritmo Cohere basata sui clic reali degli utenti.
\end{itemize}

\section{Q4: Launch \& Scale (Mesi 10-12)}
Obiettivo: Apertura al mercato, monetizzazione e scalabilità.

\subsection{Mese 10: Public Launch}
\begin{itemize}
    \item \textbf{Marketing Campaign:} Lancio ufficiale su Porfirio Magazine e canali social.
    \item \textbf{Paywall Activation:} Attivazione delle feature Premium (Concierge Mode).
    \item \textbf{City Expansion:} Apertura dei dati per Roma e Londra.
\end{itemize}

\subsection{Mese 11: B2B API Pilot}
\begin{itemize}
    \item \textbf{Documentation:} Rilascio della documentazione API per sviluppatori terzi.
    \item \textbf{Pilot Partners:} Integrazione del motore A.R.B.O.R. nel sito di un partner selezionato (e.g. Catena Hotel Lusso).
\end{itemize}

\subsection{Mese 12: Optimization & Rust Migration}
\begin{itemize}
    \item \textbf{Performance Review:} Analisi dei colli di bottiglia.
    \item \textbf{Rust Rewrite:} Inizio della migrazione dei microservizi critici da Python a Rust per ridurre i costi cloud del 50\%.
\end{itemize}

\section{Milestones Tecniche (KPIs)}

\begin{table}[h]
\centering
\begin{tabularx}{\textwidth}{|l|X|l|}
\hline
\textbf{Fase} & \textbf{Deliverable} & \textbf{Success Metric} \\
\hline
\textbf{Q1} & Database popolato (Milano) & 500+ Negozi Vetted \\
\hline
\textbf{Q2} & API Funzionante & Risposta < 2.5 sec \\
\hline
\textbf{Q3} & App Beta & Crash Rate < 0.1\% \\
\hline
\textbf{Q4} & Public Launch & 10k Monthly Active Users \\
\hline
\end{tabularx}
\caption{Key Performance Indicators per il primo anno}
\end{table}