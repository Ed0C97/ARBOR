\chapter{I Dati}
\section{Struttura Dati: Foglio \textit{Venues}}

Di seguito è riportata la definizione puntuale delle colonne presenti nel file master Excel. I campi sono divisi per competenza: \textbf{[MANUALE]} (compilati dal Curatore) e \textbf{[AI]} (generati automaticamente dagli script di arricchimento).

\subsection{Elenco Campi e Definizioni}

\begin{description}
    \item[\hl{Nome}] \texttt{[MANUALE]} \\
    Il nome ufficiale dell'attività commerciale o dell'artigiano. \\
    \textit{Facsimile: E. Marinella}

    \item[\hl{Città}] \texttt{[MANUALE]} \\
    La città di riferimento (senza indirizzo o CAP). \\
    \textit{Facsimile: Napoli}

    \item[\hl{Categoria}] \texttt{[MANUALE]} \\
    La categoria merceologica principale. I valori ammessi sono tassativi (in Inglese):
    \begin{itemize}
        \item \texttt{Accessories} (Accessori)
        \item \texttt{Books \& Music} (Libri e Musica)
        \item \texttt{Clothing} (Abbigliamento)
        \item \texttt{Food \& Drink} (Cibo e Bevande)
        \item \texttt{Footwear} (Calzature)
        \item \texttt{Fragrance \& Grooming} (Profumeria e Barba)
        \item \texttt{Motors} (Motori)
        \item \texttt{Tailoring} (Sartoria su misura)
    \end{itemize}

    \item[\hl{Note}] \texttt{[MANUALE]} \\
    Osservazioni libere del Curatore. Dettagli su esperienza, atmosfera, avvertenze. \\
    \textit{Facsimile: Negozio storico minuscolo, andare presto la mattina per evitare la folla.}

    \item[\hl{Genere}] \texttt{[MANUALE]} \\
    Il target di genere del negozio. Codici ammessi:
    \begin{itemize}
        \item \texttt{M}: Uomo (Men)
        \item \texttt{W}: Donna (Women)
        \item \texttt{MW}: Uomo e Donna
        \item \texttt{NC}: Non Classificabile / Neutro (es. Cibo, Motori)
    \end{itemize}

    \item[\hl{Stile}] \texttt{[MANUALE]} \\
    Il livello di formalità. Codici ammessi:
    \begin{itemize}
        \item \texttt{Casual}
        \item \texttt{Formal}
        \item \texttt{Casual and Formal}
        \item \texttt{Neutral} (per categorie non vestiarie)
    \end{itemize}

    \item[\hl{Link Maps}] \texttt{[MANUALE]} \\
    URL diretto alla scheda Google Maps. \\
    \textit{Facsimile: https://maps.app.goo.gl/...}

    \item[\hl{Referente}] \texttt{[MANUALE]} \\
    Nome del proprietario, store manager o commesso di fiducia da cercare. \\
    \textit{Facsimile: Maurizio}

    \item[\hl{Verificato}] \texttt{[MANUALE]} \\
    Indica se il luogo è stato visitato fisicamente dal Curatore.
    \begin{itemize}
        \item \texttt{SI}: Visitato e approvato.
        \item \texttt{No}: In lista desideri o segnalato da terzi (da verificare).
    \end{itemize}

    \item[\hl{Voto}] \texttt{[MANUALE]} \\
    Giudizio sintetico del Curatore (vedi Legenda Voto).

    \item[\hl{Founder Tags}] \texttt{[MANUALE]} \\
    Tag \textit{di pancia} che descrivono l'esperienza umana e dettagli non trovabili online. \\
    \textit{Facsimile: Istituzionale, Caotico, Top Quality, Scorbutico}

    \item[\hl{Prezzo}] \texttt{[AI]} \\
    Fascia di prezzo stimata (vedi Legenda Prezzo).

    \item[\hl{Vibe Tags}] \texttt{[AI]} \\
    Lista di 15-20 aggettivi in Inglese generati dall'AI per descrivere atmosfera e materiali. \\
    \textit{Facsimile: Heritage, Silk, Neapolitan, Bespoke, Wood-paneled}

    \item[\hl{Signature Items}] \texttt{[AI]} \\
    I 3-5 prodotti o servizi iconici per cui il luogo è famoso. \\
    \textit{Facsimile: Cravatta 7 Pieghe, Sciarpe Seta}

    \item[\hl{Target}] \texttt{[AI]} \\
    Il profilo cliente ideale (vedi Legenda Target).

    \item[\hl{Visual Style}] \texttt{[AI]} \\
    Breve descrizione dell'estetica e dell'arredamento. \\
    \textit{Facsimile: Boutique storica in legno scuro con vetrine d'epoca.}

    \item[\hl{Storia}] \texttt{[AI]} \\
    Snippet storico sulla fondazione o reputazione. \\
    \textit{Facsimile: Fondata nel 1914, fornitore ufficiale Real Casa inglese.}
\end{description}

\subsection{Legenda dei Valori Chiave}

Di seguito la spiegazione dettagliata delle scale di valutazione utilizzate.

\subsubsection{Legenda Voto (1-5)}
Il voto è espresso su una scala relativa interna alla selezione A.R.B.O.R.
Poiché il database è un \textit{Walled Garden}, la sola presenza in lista implica già il superamento della soglia di qualità. Pertanto, il voto \textbf{1} non indica insufficienza, ma il livello base di ingresso nella nostra selezione.

\begin{itemize}
    \item \textbf{1 (Valido):} Ha meritato l'ingresso in lista. Un indirizzo corretto, onesto e affidabile nel suo genere, pur senza eccellere in unicità.
    \item \textbf{2 (Buono):} Qualità superiore alla media. Un posto dove si torna volentieri, con un prodotto solido e ben eseguito.
    \item \textbf{3 (Ottimo):} Il \textit{Gold Standard}. Eccellenza tecnica e stilistica. È il punto di riferimento per la sua categoria.
    \item \textbf{4 (Eccellente):} Esperienza memorabile. Oltre al prodotto perfetto, c'è un fattore \textit{X} (storia, atmosfera, servizio) che lo rende speciale.
    \item \textbf{5 (Capolavoro / Icona):} L'apice assoluto. Un'istituzione sacra, un luogo di pellegrinaggio per gli intenditori. Perfezione indiscutibile.
\end{itemize}

\subsubsection{Legenda Prezzo (1-5)}
Indica il posizionamento economico rispetto alla media di mercato della categoria.
\begin{itemize}
    \item \textbf{1 (Economico):} Prezzi bassi, affari, street food economico.
    \item \textbf{2 (Accessibile):} Prezzi medi, buon rapporto qualità/prezzo.
    \item \textbf{3 (Premium):} Fascia alta ma standard. Costoso ma giustificato.
    \item \textbf{4 (Lusso):} Prezzi elevati, brand prestigiosi, materiali pregiati.
    \item \textbf{5 (Ultra-Lusso / Bespoke):} Prezzi senza limite. Alta sartoria, gioielleria, servizi esclusivi su misura.
\end{itemize}

\subsubsection{Legenda Target (Profilazione AI)}
Indica a quale tipologia di utente è più adatto il luogo, per guidare il Router dell'AI.
\begin{itemize}
    \item \textbf{Expert Only:} Luoghi intimidatori, nascosti o tecnici. Spesso su appuntamento, senza vetrina o con barriere linguistiche/culturali. Richiedono competenza per essere apprezzati (e per non essere trattati male).
    \item \textbf{Enthusiast:} Luoghi di alta qualità e passione. Accoglienti ma specifici. Richiedono un interesse per la materia, ma il personale è disposto a spiegare.
    \item \textbf{Tourist Friendly:} Luoghi facili, centrali, con personale che parla inglese e servizi Tax Free. Esperienza d'acquisto fluida e senza stress.
    \item \textbf{Local Gem:} Istituzioni di quartiere. Non necessariamente lusso, ma autentici e frequentati dai residenti. Il posto \textit{vero}.
    \item \textbf{High Spender:} Luoghi focalizzati sul VIP treatment. Lusso sfrenato, servizio impeccabile, ambiente esclusivo. Adatto a chi vuole spendere per essere coccolato.
\end{itemize}

\section{Struttura Dati: Foglio Brands}

Questa tabella gestisce le entità astratte (i Marchi). A differenza dei Negozi (luoghi fisici), i Brand rappresentano concetti, storia e prodotti che possono essere venduti in più luoghi.

\subsection{Elenco Campi e Definizioni}

\begin{description}
    \item[\hl{Nome}] \texttt{[MANUALE]} \\
    Il nome ufficiale del Brand o dell'Azienda produttrice. \\
    \textit{Facsimile: Barbour}

    \item[\hl{Categoria}] \texttt{[MANUALE]} \\
    La categoria merceologica principale (stessa tassonomia di Venues). \\
    \textit{Facsimile: Clothing}

    \item[\hl{Rivenditori}] \texttt{[MANUALE] - \textbf{CRITICO}} \\
    Elenco dei negozi fisici (presenti nel foglio Venues) che vendono questo brand. I nomi devono corrispondere esattamente. Separare con virgola. \\
    \textit{Facsimile: Davide Cenci, WP Store, Barbour Store Roma}

    \item[\hl{Note}] \texttt{[MANUALE]} \\
    Opinione del Curatore sul brand (evoluzione della qualità, reputazione attuale). \\
    \textit{Facsimile: Qualità del cotone cerato calata negli anni, ma resta un'icona indistruttibile.}

    \item[\hl{Link Sito}] \texttt{[MANUALE]} \\
    Sito web ufficiale del brand. \\
    \textit{Facsimile: www.barbour.com}

    \item[\hl{Founder Tags}] \texttt{[MANUALE]} \\
    Tag di pancia sulla percezione del brand. \\
    \textit{Facsimile: Indistruttibile, British, Inflazionato, Outdoor}

    \item[\hl{Nazione}] \texttt{[AI]} \\
    Paese di origine o sede principale del brand. \\
    \textit{Facsimile: Regno Unito}

    \item[\hl{Prezzo}] \texttt{[AI]} \\
    Posizionamento di prezzo medio del brand (1-5).

    \item[\hl{Vibe Tags}] \texttt{[AI]} \\
    Nuvola di aggettivi che descrivono l'estetica del brand. \\
    \textit{Facsimile: Countryside, Rainy, Waxed Cotton, Hunting, Royal Warrant}

    \item[\hl{Signature Items}] \texttt{[AI]} \\
    I prodotti più celebri prodotti dal brand. \\
    \textit{Facsimile: Bedale Jacket, Beaufort, Tartan Scarf}

    \item[\hl{Target}] \texttt{[AI]} \\
    A chi si rivolge il brand (stessa legenda di Venues). \\
    \textit{Facsimile: Enthusiast}

    \item[\hl{Stile}] \texttt{[AI]} \\
    Definizione sintetica dello stile. \\
    \textit{Facsimile: English Country}

    \item[\hl{Storia}] \texttt{[AI]} \\
    Snippet storico sulla fondazione. \\
    \textit{Facsimile: Fondata nel 1894 a South Shields, famosa per le giacche cerate.}
\end{description}

\section{Data Interconnection: La Genesi del Grafo}

Il vero valore di A.R.B.O.R. risiede nella capacità di collegare le due tabelle (\textit{Venues} e \textit{Brands}) per creare un **Knowledge Graph** navigabile.

\subsection{Il Ponte Logico: La Colonna Rivenditori}
La connessione avviene tramite la colonna \texttt{Rivenditori} nel foglio Brands.
Lo script di ingestione (\texttt{ingest\_master.py}) esegue la seguente logica di collegamento:

\begin{enumerate}
    \item Legge il nome del Brand (es. \textbf{Barbour}).
    \item Legge la lista dei Rivenditori (es. \textbf{Davide Cenci}).
    \item Cerca nel database dei Negozi se esiste un nodo chiamato Davide Cenci.
    \item Se esiste, crea una relazione direzionale nel grafo Neo4j.
\end{enumerate}

\subsection{Tipologie di Relazioni (Graph Edges)}
Il sistema è abbastanza intelligente da distinguere il tipo di relazione basandosi sui \textit{Founder Tags} del negozio.

\begin{itemize}
    \item \textbf{Relazione Standard (SELLS):}
    Se il negozio è un multimarca.
    \[ (:Venue \text{ Davide Cenci}) \xrightarrow{\text{SELLS\_BRAND}} (:Brand \text{ Barbour}) \]
    \textit{Significato:} Qui puoi comprare questo brand.

    \item \textbf{Relazione HQ (IS\_HQ\_OF):}
    Se il negozio ha nei Founder Tags parole come Flagship, Factory o Monobrand.
    \[ (:Venue \text{ Barbour Store Roma}) \xrightarrow{\text{IS\_HQ\_OF}} (:Brand \text{ Barbour}) \]
    \textit{Significato:} Questa è la casa madre o la rappresentanza ufficiale del brand.
\end{itemize}

\subsection{Vantaggio per l'Utente}
Questa struttura permette all'AI di rispondere a domande complesse di secondo livello:
\begin{quote}
    \textit{Mi piace lo stile di Barbour (Brand), ma sono a Roma. Portami in un negozio (Venue) che abbia quella stessa atmosfera (Vibe), anche se vende altri marchi.}
\end{quote}
Il sistema naviga dal Brand ai suoi Rivenditori, analizza il Vibe dei Rivenditori e trova luoghi simili vettorialmente.
