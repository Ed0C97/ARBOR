\documentclass[11pt, a4paper]{report}

% Pacchetti fondamentali
\usepackage[utf8]{inputenc}
\usepackage[T1]{fontenc}
\usepackage[italian]{babel}
\usepackage{geometry}
\usepackage{titlesec}
\usepackage{hyperref}
\usepackage{graphicx}
\usepackage{xcolor}
\usepackage{listings}
\usepackage{booktabs}
\usepackage{longtable}
\usepackage{array}
\usepackage{tabularx}
\usepackage{pdfpages}
\usepackage{pdflscape}
\usepackage{amsmath}
\usepackage{soul,xcolor}
\sethlcolor{yellow}

% Configurazione margini
\geometry{
    top=2.5cm,
    bottom=2.5cm,
    left=2.5cm,
    right=2.5cm
}

% Configurazione Link
\hypersetup{
    colorlinks=true,
    linkcolor=darkgray,
    urlcolor=blue,
    pdftitle={Business Plan \& Technical Architecture - The Contextual Engine},
    pdfpagemode=FullScreen,
}

% Configurazione Code Blocks
\definecolor{codegray}{rgb}{0.95,0.95,0.95}
\definecolor{commentgreen}{rgb}{0,0.6,0}
\lstset{
    backgroundcolor=\color{codegray},
    basicstyle=\ttfamily\small,
    breaklines=true,
    captionpos=b,
    commentstyle=\color{commentgreen},
    keywordstyle=\color{blue},
    stringstyle=\color{red},
    frame=single,
    showstringspaces=false
}

% Titolo
\title{
    \vspace{-3cm} % Tira su il tutto per fare spazio
    
    % --- SEZIONE LOGO ---
    % Sostituisci 'logo.png' con il nome del tuo file.
    % Cambia 'width=5cm' per ingrandirlo o rimpicciolirlo.
    \begin{center}
        \includegraphics[width=5cm]{ARBOR_logo.png} 
    \end{center}
    \vspace{1cm} % Spazio tra logo e testo
    % --------------------

    {\scshape \small Porfirio Holding Presents} \\ \vspace{0.3cm}
    \rule{\linewidth}{0.5mm} \\ \vspace{0.8cm}
    
    % IL NOME DEL PROGETTO
    \textbf{\Huge PROJECT A.R.B.O.R.} \\ \vspace{0.4cm}
    
    % L'ACRONIMO SCIOLTO (Tecnico \& Filosofico)
    {\LARGE Advanced Reasoning By Ontological Rules} \\ \vspace{0.8cm}
    
    % LA DEFINIZIONE DI PRODOTTO
    {\Large \textbf{The Contextual Discovery Engine}} \\ \vspace{1cm}
    
    % IL MOTTO (La Visione)
    ``Digitalizing the Instinct of Quality.'' \\ \vspace{0.8cm}
    
    \rule{\linewidth}{0.5mm} \\ \vspace{0.5cm}
    
    \small Documento di Strategia Industriale \& Architettura Tecnica
}
\author{Edited for Porfirio’s Magazine, by appointment to Porfirio Holding Ltd.}
\date{\today}

\begin{document}

\maketitle

\begin{abstract}
Questo documento descrive la strategia, l'architettura tecnica e il modello di business per lo sviluppo di una piattaforma di Curated AI Discovery. Il sistema supera i limiti dei tradizionali motori di ricerca (Google Maps) e degli algoritmi di raccomandazione collaborativi (Amazon/Netflix), introducendo un modello basato su Semantic Vector Graphs e Dimensional Vibe Scoring. L'obiettivo è digitalizzare l'esperienza e il contesto (Vibe) per offrire raccomandazioni cross-domain di alta precisione.
\end{abstract}

\tableofcontents
\clearpage

% =========================
% Capitoli principali
% =========================
\chapter{Executive Summary e Vision Strategica}

\section{L'Era della \textit{Curated Discovery}}

Viviamo in un momento storico definito dall'\textbf{Information Overload}. Negli ultimi vent'anni, la missione delle grandi piattaforme tecnologiche (Google, Yelp, TripAdvisor) è stata quella di aggregare e indicizzare la totalità delle informazioni mondiali. Hanno avuto successo: oggi è tecnicamente possibile trovare qualsiasi negozio, ristorante o servizio in pochi millisecondi.
Tuttavia, l'accesso ubiquo all'informazione ha generato un nuovo problema critico: il \textbf{Rumore dei Dati (Data Noise)}.
L'utente moderno non ha più difficoltà a trovare opzioni; ha difficoltà a filtrare opzioni valide. La democratizzazione della recensione (User Generated Content) ha livellato verso il basso la qualità del segnale: quando tutti sono esperti, nessuno lo è. Il rapporto Segnale/Rumore è drasticamente peggiorato.
Il nostro progetto nasce per guidare la transizione dalla fase di \textbf{Search} (Ricerca passiva basata su parole chiave) alla fase di \textbf{Discovery} (Scoperta attiva guidata dal contesto). Non stiamo costruendo un motore di ricerca più grande; stiamo costruendo un filtro più intelligente. L'obiettivo è creare l'infrastruttura tecnologica per la digitalizzazione del gusto e della competenza, superando i limiti degli attuali aggregatori generalisti.

\section{Il Problema: Il Fallimento degli Standard Attuali}

L'attuale ecosistema digitale presenta tre fallimenti strutturali (Market Failures) che rendono impossibile una ricerca di qualità per l'utente esigente:

\subsection{La Tirannia della Media (Il Problema delle 5 Stelle)}
Il sistema di rating universale (scala 1-5 stelle) è matematicamente fallace quando applicato a esperienze soggettive, qualitative e di nicchia.
\begin{itemize}
\item \textbf{Mancanza di Contesto:} Un fast-food economico può avere 4.8 stelle perché è ottimo per essere un fast-food. Una sartoria di alto livello può avere 3.9 stelle perché un utente inesperto si è lamentato del prezzo o dei tempi di attesa, fattori che sono invece intrinseci all'alta qualità artigianale. \item \textbf{Distorsione Algoritmica:} Agli occhi di un algoritmo classico (e.g. Google Maps sorting), il fast-food appare migliore della sartoria. L'utente alla ricerca di eccellenza viene attivamente disinformato dalla piattaforma, che premia la popolarità rispetto alla qualità.
\end{itemize}

\subsection{L'Ottimizzazione SEO vs. La Qualità Reale}
I risultati di ricerca odierni (SERP\footnote{\textbf{Search Engine Results Page}: è la pagina web visualizzata da un motore di ricerca in risposta a una query dell'utente. Include sia i risultati organici che gli annunci a pagamento.}) premiano le aziende che investono in marketing digitale, non necessariamente quelle che investono nel prodotto.
\begin{itemize}
\item Un'attività mediocre con un ottimo consulente SEO apparirà sempre posizionata meglio di una bottega storica eccellente che non possiede un sito web ottimizzato.
\item Questo crea una \textbf{asimmetria informativa}: la visibilità digitale è slegata dal merito reale.
\end{itemize}

\subsection{La Cecità Semantica: Keyword vs. Intent}
I motori di ricerca attuali ragionano prevalentemente per sintassi (parole chiave), non per semantica (significati profondi).
\begin{itemize}
\item Esempio: Se un utente cerca \textit{Cravatta sportiva}, un motore classico cerca la stringa di testo \textit{sportiva}. Il risultato includerà cravatte in poliestere con loghi di squadre di calcio.
\item Realtà: L'utente esperto, per \textit{sportiva}, intende concetti complessi come maglia, tricot, sfoderata, garza di seta, shantung.
\item Gap Tecnologico: Attualmente non esiste un sistema capace di colmare il divario tra il \textbf{linguaggio naturale dell'utente} e il \textbf{linguaggio tecnico del dominio} senza richiedere all'utente di diventare un esperto di query booleane.
\end{itemize}

\section{La Soluzione: \textit{The Contextual Engine}}

Proponiamo una piattaforma di \textbf{Discovery Verticale e Curata}, potenziata da un'architettura ibrida Uomo-AI (Human-in-the-Loop). Il sistema non mira a mappare il mondo intero, ma a mappare esclusivamente l'eccellenza, decodificandola attraverso parametri dimensionali complessi.
La soluzione poggia su tre pilastri tecnologici e metodologici:

\begin{enumerate}
\item \textbf{Validazione a Monte - The Walled Garden:}
A differenza degli aggregatori aperti, il nostro database è un ecosistema chiuso. Ogni entità (negozio, brand, servizio) deve superare un protocollo di validazione (\textbf{Human-Vetted}). Questo garantisce che ogni risultato di ricerca sia, per definizione, pertinente. Eliminiamo alla radice il rischio di falsi positivi.

\item \textbf{Dimensional Vibe Scoring:}
Sostituiamo il voto lineare (Stella) con un profilo vettoriale multidimensionale. Ogni entità viene analizzata e taggata su assi specifici (e.g. Formalità, Artigianalità, Rapporto Qualità/Prezzo, Atmosfera, Esclusività). 
Questo permette all'utente di modulare la ricerca con granularità fine: \textit{Voglio un luogo con alta artigianalità }(Score > 80)\textit{, ma bassa formalità }(Score < 40).

\item \textbf{Cross-Domain Transitivity:}
Utilizzando un \textbf{Knowledge Graph Semantico}, il sistema riconosce pattern di gusto trasversali. L'algoritmo comprende che l'estetica e i valori di un determinato Hotel di design a Tokyo condividono lo stesso DNA di una specifica torrefazione a Copenaghen. Questo abilita raccomandazioni che attraversano le categorie merceologiche, cosa impossibile per gli algoritmi di Collaborative Filtering (e.g. Amazon) che lavorano per compartimenti stagni.
\end{enumerate}

\section{Mission e Vision}

\begin{description}
\item[Mission] Restituire valore al tempo dell'utente e dignità all'eccellenza commerciale, creando la connessione più breve, precisa e fidata tra una domanda complessa (Intent) e la soluzione perfetta (Asset).
\item[Vision] Diventare il \textbf{protocollo standard globale} per la ricerca basata sul contesto (\textit{Context-First Search}). L'obiettivo a lungo termine è evolvere da una guida verticale (Lifestyle/Shopping) a un'infrastruttura SaaS (Software as a Service) applicabile a settori critici come Real Estate, Hospitality e Human Resources, dove il \textit{Cultural Fit} è più importante dei dati grezzi.
\end{description}

\section{Unique Value Proposition}

Per l'utente finale, il valore è riassumibile in tre concetti chiave:

\begin{itemize}
\item \textbf{Trust:} La certezza matematica che il risultato non è frutto di una sponsorizzazione nascosta (ADS) o di una manipolazione algoritmica SEO.
\item \textbf{Efficiency:} Riduzione drastica del tempo di ricerca. Il processo passa da ore di confronto su forum, blog e recensioni contrastanti a pochi secondi di interazione in linguaggio naturale.
\item \textbf{Mentorship:} Il sistema non fornisce solo il \textit{dove} (Coordinate), ma spiega il \textit{perché} (Contesto), educando l'utente alla qualità e affinando il suo gusto nel tempo attraverso spiegazioni generate dall'AI.
\end{itemize}
\chapter{Analisi di Mercato e Scenario Competitivo}

\section{Il Nuovo Paradigma: Dalla \textit{Attention Economy} alla \textit{Trust Economy}}

Il mercato digitale sta attraversando una fase di transizione tettonica. Nell'ultimo decennio, il modello dominante è stato quello dell'Attention Economy: le piattaforme competevano per massimizzare il tempo speso dall'utente sullo schermo, spesso a scapito della qualità dell'informazione. Questo modello ha saturato l'utente, generando diffidenza.
Oggi assistiamo all'emergere della \textbf{Trust Economy} (Economia della Fiducia). Gli utenti alto-spendenti (High Net Worth Individuals - HNWI) e i consumatori consapevoli stanno abbandonando le piattaforme generaliste in favore di ecosistemi curati, verticali e privi di rumore.
La disponibilità a pagare (Willingness to Pay) si sta spostando dall'accesso al contenuto (che è commodity) al \textbf{filtro sul contenuto} (che è premium). Il nostro progetto si posiziona esattamente in questa nicchia di mercato in espansione: la fornitura di certezze in un mare di probabilità.

\section{Analisi Tecnica dei Competitor}

Per comprendere il posizionamento strategico del Contextual Discovery Engine, è necessario analizzare le limitazioni architetturali degli attuali player di mercato. Non si tratta solo di differenze di brand, ma di differenze strutturali nel modo in cui i dati vengono processati.

\subsection{Google Maps / TripAdvisor (Generalisti)}
\begin{itemize}
    \item \textbf{Modello Dati:} Crowd-Sourced. Chiunque può scrivere, chiunque può alterare la media.
    \item \textbf{Incentivo:} Volume. Il modello di business si basa sulle Ads. Più tempo passi a cercare, più pubblicità vedi. Non hanno incentivo a darti la risposta giusta al primo colpo.
    \item \textbf{Limite Tecnico:} \textit{Keyword Matching}. Ottimi per trovare \textit{Benzinaio aperto}, pessimi per trovare \textit{Sartoria con taglio inglese morbido}.
\end{itemize}

\subsection{Instagram / TikTok (Social Discovery)}
\begin{itemize}
    \item \textbf{Modello Dati:} Visual \& Viral. L'algoritmo premia ciò che è esteticamente appagante o controverso, non ciò che è qualitativamente valido.
    \item \textbf{Incentivo:} Engagement. L'obiettivo è trattenere l'utente nel feed, non mandarlo nel negozio fisico.
    \item \textbf{Limite Tecnico:} \textit{Hidden ADS}. La pervasività delle sponsorizzazioni non dichiarate (Influencer Marketing) ha eroso la fiducia. L'utente non sa mai se un consiglio è genuino.
\end{itemize}

\subsection{Amazon / Netflix (Algoritmi Collaborativi)}
Questi giganti eccellono nel consigliare prodotti di massa, ma falliscono nel Lifestyle Discovery. La ragione è matematica.

\begin{table}[h]
\centering
\renewcommand{\arraystretch}{1.5} % Aumenta lo spazio tra le righe per leggibilità
% Usa tabularx per adattare la tabella alla larghezza del testo (\textwidth)
% |X|X| significa: due colonne di larghezza uguale che vanno a capo automaticamente
\begin{tabularx}{\textwidth}{|X|X|}
\hline
\textbf{Collaborative Filtering \newline (Amazon/Netflix)} & \textbf{Semantic Vector Graph \newline (La Nostra Soluzione)} \\
\hline
\textbf{Logica:} \textit{Chi ha comprato X ha comprato anche Y}. Si basa sulla statistica di massa. & \textbf{Logica:} \textit{L'oggetto X ha lo stesso DNA dell'oggetto Y}. Si basa sull'analisi intrinseca. \\
\hline
\textbf{Problema \textit{Cold Start}:} Se un negozio è nuovo o di nicchia e ha pochi clic, l'algoritmo lo ignora, anche se è eccellente. & \textbf{Soluzione Zero-Shot:} Un negozio viene raccomandato dal Giorno 1 se il suo \textit{Vibe Score} matcha la richiesta, indipendentemente dalla popolarità. \\
\hline
\textbf{Silos Verticali:} Non sa correlare domini diversi (e.g. un Vino e una Giacca) perché mancano dati transazionali congiunti. & \textbf{Cross-Domain:} Collega settori diversi tramite \textit{nodi di atmosfera} (e.g. \texttt{Artigianale}, \texttt{Minimal}, \texttt{Barocco}). \\
\hline
\textbf{Risultato:} Crea \textit{Bolle di Filtro} (ti mostra sempre lo stesso). & \textbf{Risultato:} Abilita la \textit{serendipity} (scoperta inaspettata ma pertinente). \\
\hline
\end{tabularx}
\caption{Confronto Architetturale: Collaborative vs Semantic Filtering}
\label{tab:comparison}
\end{table}

\section{Target Audience}

Il nostro utente tipo non è definito dall'età, ma dall'attitudine verso il consumo e il tempo. Abbiamo identificato tre profili psicografici primari.

\subsection{Persona A: The Quality Seeker (Il Purista)}
\begin{itemize}
    \item \textbf{Profilo:} Appassionato, alto-spendete, esteta. Legge forum specializzati, odia il mainstream.
    \item \textbf{Pain Point:} Perde ore a fare \textit{Fact Checking} sulle recensioni per evitare trappole per turisti.
    \item \textbf{Uso della Piattaforma:} Cerca validazione. Usa il sistema come un'autorità di certificazione. Vuole dettagli tecnici (\textit{La giacca è intelata o adesivata?}).
\end{itemize}

\subsection{Persona B: The Business Traveler (Il Pragmatico)}
\begin{itemize}
    \item \textbf{Profilo:} Professionista in viaggio, budget alto, tempo zero. Ha 2 ore libere tra un meeting e l'aeroporto.
    \item \textbf{Pain Point:} Inefficienza. Non può permettersi di girare a vuoto. Vuole andare a colpo sicuro.
    \item \textbf{Uso della Piattaforma:} Cerca efficienza logistica unita alla qualità. \textit{Sono qui, ho 45 minuti, voglio comprare un regalo per mia moglie che non sia banale}.
\end{itemize}

\subsection{Persona C: The Aspiring Connoisseur (Il Neofita)}
\begin{itemize}
    \item \textbf{Profilo:} Giovane professionista che vuole elevare il proprio stile di vita ma non ha ancora le competenze (o il vocabolario) per giudicare da solo.
    \item \textbf{Pain Point:} Paura sociale (Social Anxiety). Teme di entrare nel negozio sbagliato, di essere giudicato o di spendere male.
    \item \textbf{Uso della Piattaforma:} Cerca mentorship. Usa l'AI come una guida sicura che gli spiega i termini e lo indirizza verso luoghi accessibili al suo livello di comfort.
\end{itemize}

\section{Market Trends \& Timing (Perché Ora?)}

Il lancio di questo progetto intercetta tre macro-trend globali che rendono il momento storico perfetto (Market Timing):

\begin{enumerate}
    \item \textbf{L'Ascesa del \textit{Quiet Luxury}:} Dopo anni di logomania e ostentazione, il mercato del lusso si sta spostando verso l'understatement, la qualità dei materiali e l'artigianato nascosto. Questi sono asset che Google non sa indicizzare, ma il nostro Contextual Engine sì.
    \item \textbf{La Crisi dell'Influencer Marketing:} I consumatori sono saturi di contenuti sponsorizzati. C'è una fame disperata di voci autorevoli, imparziali e tecniche.
    \item \textbf{L'Adozione dell'AI Conversazionale:} Grazie a ChatGPT, l'utente è ora abituato a interagire con il software via chat (Natural Language Processing) invece che cliccare su filtri statici. La UX del nostro prodotto è ora nativa per la maggioranza degli utenti.
\end{enumerate}

\section{Barriere all'Ingresso (Difendibilità)}

Perché un gigante non può copiarci domani? (The Moat)

\begin{itemize}
    \item \textbf{Data Proprietary Moat:} L'AI è una commodity, i dati no. Mentre i competitor addestrano i modelli su dati pubblici rumorosi, noi costruiamo un dataset proprietario curato. Un modello AI è intelligente solo quanto i dati su cui ragiona.
    \item \textbf{Human-in-the-Loop Protocol:} La nostra infrastruttura prevede la validazione umana esperta. Per replicarla, Google dovrebbe assumere migliaia di esperti di stile locali, un modello operativo non scalabile per la loro struttura generalista.
    \item \textbf{Vibe Ontology:} L'ontologia proprietaria (il modo in cui classifichiamo le \textit{vibes}) è proprietà intellettuale. Abbiamo definito uno standard di codifica dell'esperienza che funge da protocollo unico.
\end{itemize}
\chapter{The A.R.B.O.R. Architecture: God Mode}

\section{Overview: The Cognitive Stack}

A.R.B.O.R. non è una semplice applicazione web, ma una \textbf{Cognitive Architecture} distribuita che supera lo stato dell'arte attuale. Abbandoniamo l'approccio monolitico per adottare un sistema a \textbf{Sciame di Agenti (Agentic Swarm)} supportato da una \textbf{Trinità di Dati}.

Il sistema è progettato per gestire due flussi temporali:
\begin{itemize}
    \item \textbf{Sincrono (Real-Time):} Risposta all'utente in $<2$ secondi tramite Cache e Vettori.
    \item \textbf{Asincrono (Background):} Ingestione e ragionamento profondo tramite Grafo e LLM.
\end{itemize}

% Inizia la pagina orizzontale
\begin{landscape}
    \begin{figure}[p] % [p] forza la figura su una pagina dedicata senza testo
        \centering
        % Impostiamo larghezza e altezza massime mantenendo le proporzioni
        \includegraphics[width=\linewidth, height=1\textheight, keepaspectratio]{grafico_arbor.pdf}
        \caption{Master Blueprint: Architettura GraphRAG + Agentic Swarm}
        \label{fig:blueprint}
    \end{figure}
\end{landscape}
% Finisce la pagina orizzontale e torna verticale

\section{Layer 1: The Knowledge Trinity (La Memoria)}

Per permettere un ragionamento \textit{umano}, il sistema utilizza tre tipologie di database simultaneamente.

\paragraph{1. PostgreSQL (The Source of Truth)}
Contiene i fatti oggettivi e immutabili. Garantisce l'integrità transazionale (ACID).
\begin{itemize}
    \item \textbf{Dati:} Anagrafiche, Indirizzi, Prezzi, Orari.
    \item \textbf{Tecnologia:} PostgreSQL 16 con estensione PostGIS per la geolocalizzazione di precisione.
\end{itemize}

\paragraph{2. Qdrant (The Intuition)}
Il motore vettoriale scritto in Rust. Gestisce la \textit{ricerca per sensazione} (Vibe).
\begin{itemize}
    \item \textbf{Dati:} Embeddings (vettori a 1536 dimensioni) delle descrizioni e delle immagini.
    \item \textbf{Funzione:} Permette query come \textit{Trova un posto con atmosfera simile a questo}, impossibile per i database classici.
\end{itemize}

\paragraph{3. Neo4j (The Logic)}
Il Knowledge Graph. Mappa le relazioni invisibili e storiche.
\begin{itemize}
    \item \textbf{Dati:} Nodi (Persone, Brand, Stili) e Archi (Relazioni: \texttt{TRAINED\_BY}, \texttt{INSPIRED\_BY}).
    \item \textbf{Funzione:} Abilita il ragionamento transitivo: \textit{Consigliami questo sarto perché il suo maestro ha lavorato per Marinella}.
\end{itemize}

\section{Layer 2: The Agentic Swarm (Il Ragionamento)}

L'orchestrazione è gestita da \textbf{LangGraph}, che coordina agenti specializzati:

\begin{enumerate}
    \item \textbf{Intent Router:} Classifica la richiesta dell'utente e attiva l'agente giusto.
    \item \textbf{Vector Agent:} Interroga Qdrant per similarità estetica.
    \item \textbf{Historian Agent:} Interroga Neo4j per connessioni storiche.
    \item \textbf{Metadata Agent:} Interroga Postgres per filtri rigidi (prezzo, apertura).
    \item \textbf{The Curator:} Sintetizza i risultati e genera la risposta finale con il tono di voce del brand.
\end{enumerate}

\section{Layer 3: Ingestion \& Human-in-the-Loop}

La qualità del dato è garantita da una pipeline ibrida AI-Umana.
\begin{enumerate}
    \item \textbf{Scraper \& Vision AI:} Scaricano dati e analizzano foto per generare bozze.
    \item \textbf{Curator Dashboard:} Un'interfaccia dove gli esperti umani validano i dati prima della pubblicazione. Nessun dato entra nel sistema senza il marchio \textit{Vetted}.
\end{enumerate}

\section{Stack Tecnologico: Hybrid Performance}

Adottiamo una strategia \textit{Brain \& Muscle}:
\begin{itemize}
    \item \textbf{Python (Brain):} Usato per la logica AI (FastAPI, LangGraph) per la massima flessibilità.
    \item \textbf{Rust (Muscle):} Usato per i motori di calcolo pesanti (Qdrant, Pydantic Core) per la massima velocità e risparmio costi cloud.
\end{itemize}
\chapter{Business Model e Go-to-Market Strategy}

\section{Posizionamento Strategico: Il \textit{Modello Michelin }Digitale}

A.R.B.O.R. rifiuta il modello \textit{Crowd-Sourced} (TripAdvisor/Google Maps) basato sulla media delle opinioni di massa, spesso incompetenti o manipolabili.
Adottiamo invece il \textbf{Modello Curatoriale Autorevole} (stile Guida Michelin), adattato all'era dell'Intelligenza Artificiale.

\subsection{I Pilastri dell'Autorità}

\begin{itemize}
    \item \textbf{Indipendenza Radicale (No Pay-to-Play):} L'inclusione nel database non è in vendita. Un brand o un negozio non può pagare per essere listato o per migliorare il proprio ranking. L'unica valuta accettata è la qualità oggettiva e il rispetto dei parametri del nostro \textit{Vibe DNA}.
    \item \textbf{Anonimato e Verifica:} Proprio come gli ispettori Michelin, il processo di validazione (\textit{Founder Tags}) avviene in modo discreto e basato sull'esperienza reale, non su cartelle stampa o inviti PR.
    \item \textbf{Scarsità del Dato:} Il valore di A.R.B.O.R. non risiede nella quantità di risultati (copertura totale), ma nella loro selezione. L'assenza di un negozio famoso dal nostro database è un segnale forte quanto la presenza di una gemma nascosta.
\end{itemize}

\subsection{Il Sistema di Classificazione (The Porfirio Standard)}

Il sistema non si limita a elencare, ma classifica l'eccellenza su tre livelli gerarchici:

\begin{enumerate}
    \item \textbf{Vetted (Verificato):} Il luogo rispetta gli standard di qualità, onestà e servizio. È un indirizzo sicuro (\textit{Safe Choice}).
    \item \textbf{Selected (Segnalato):} Il luogo possiede un carattere distintivo, un \textit{Vibe} particolare o un prodotto signature che lo eleva sopra la media.
    \item \textbf{Icon (Cult):} I templi dello stile. Luoghi che possiedono un'aura storica o estetica unica al mondo. \textit{Vale il viaggio}.
\end{enumerate}

\section{Filosofia di Monetizzazione: The Trust Currency}

Il modello di business tradizionale dei media digitali si basa sulla vendita dell'attenzione (CPM\footnote{\textbf{Cost Per Mille}: Costo per mille impressioni.}). Questo modello crea un conflitto di interessi: la piattaforma vuole che l'utente rimanga online, l'utente vuole risolvere il problema e uscire.

Il Contextual Discovery Engine adotta un approccio diametralmente opposto: monetizziamo la \textbf{risoluzione del problema}. Poiché il nostro asset principale è la \textbf{fiducia}, non possiamo accettare denaro per alterare il ranking (niente \textit{Sponsored Listings} non dichiarate).

\section{Revenue Streams (Flussi di Ricavo)}

\subsection{1. Performance Marketing \& Lead Generation (B2B)}
A differenza di Google ADS (dove si paga per il clic, spesso a vuoto), noi operiamo su un modello a \textbf{Revenue Share}.
\begin{itemize}
    \item \textbf{Modello:} Commissione sul transato (CPA).
    \item \textbf{Meccanismo:} Quando un utente scopre un prodotto tramite la nostra piattaforma e finalizza l'acquisto, percepiamo una commissione variabile (media 8\%).
    \item \textbf{Vantaggio:} Il brand paga solo a risultato ottenuto. Traffico \textit{High-Intent}.
\end{itemize}

\subsection{2. Membership Premium \textit{The Concierge} (B2C)}
Un modello Freemium per l'utente finale.
\begin{itemize}
    \item \textbf{Livello Free:} Ricerca base limitata (e.g. 3 ricerche al giorno).
    \item \textbf{Livello Curator (€14.99/mese):} Ricerche illimitate, accesso alla cronologia, itinerari personalizzati e accesso ai \textit{Perks} esclusivi (sconti/trattamenti).
\end{itemize}

\section{Go-to-Market Strategy (GTM)}

Rifiutiamo la classica strategia di \textit{saturazione geografica}. Il nostro approccio è quello della \textbf{Organic Quality Expansion}.

\paragraph{Fase 1: The Founder's Edit (Il Nucleo Curato)}
Il lancio iniziale non è definito da confini geografici, ma dall'esperienza diretta. Roma funge da quartier generale, ma il database include fin dal primo giorno le \textit{gemme} visitate personalmente in altre città (e.g. Napoli, Londra, Milano). Solo luoghi testati e validati personalmente.

\paragraph{Fase 2: Signal Scouting (Espansione Verificata)}
Per scalare, attiviamo un meccanismo di segnalazione qualificato. Gli utenti possono suggerire nuovi luoghi, ma l'inserimento avviene solo dopo una \textit{Cross-Validation} da parte del team centrale.

\section{Piano Finanziario Dettagliato (P\&L Anno 1)}

Il seguente prospetto analizza i costi vivi per mantenere l'infrastruttura \textit{God Mode} e le proiezioni di ricavo basate su un modello B2B a commissione variabile.

\subsection{Struttura dei Costi (OPEX \& CAPEX)}

\begin{table}[h]
\centering
\footnotesize
\renewcommand{\arraystretch}{1.4}
\begin{tabularx}{\textwidth}{|l|l|r|X|}
\hline
\textbf{VOCE DI SPESA} & \textbf{CATEGORIA} & \textbf{COSTO MENSILE} & \textbf{NOTE TECNICHE} \\
\hline
\multicolumn{4}{|c|}{\textbf{INFRASTRUTTURA FISSA (Core Stack)}} \\
\hline
\textbf{Sviluppatore Backend} & Personale & € 0,00 & Founder Equity (Costo opportunità: €4k/mese). \\
\hline
\textbf{Vercel Pro} & Frontend Host & € 18,00 & Hosting Next.js + Serverless Functions. \\
\hline
\textbf{Supabase Pro} & Database SQL & € 23,00 & PostgreSQL 8GB + Auth. \\
\hline
\textbf{Qdrant Cloud} & Vector DB & € 25,00 & Cluster Rust gestito (fino a 1M vettori). \\
\hline
\textbf{Neo4j AuraDB} & Graph DB & € 60,00 & Istanza Professional per il Grafo. \\
\hline
\textbf{Railway / AWS} & Backend Host & € 15,00 & Container Docker Python sempre acceso. \\
\hline
\textbf{Redis Cloud} & Caching & € 10,00 & Cache semantica. \\
\hline
\textit{Subtotale Fisso} & & \textit{€ 151,00} & Costo infrastruttura base. \\
\hline
\multicolumn{4}{|c|}{\textbf{COSTI VARIABILI (AI \& API)}} \\
\hline
\textbf{OpenAI (GPT-4o)} & Intelligence & € 0,04 / query & Stima input/output complessi. \\
\hline
\textbf{Google Maps API} & Dati Geo & € 0,00 & Coperto dal credito free mensile (\$200). \\
\hline
\textbf{Cohere Rerank} & Ranking & € 1,00 / 1k search & Riordino risultati. \\
\hline
\end{tabularx}
\caption{Tabella Costi Operativi Reali (Anno 1)}
\end{table}

\subsection{Proiezione Ricavi (Revenue Model)}

\textbf{Assunzioni:} Scontrino Medio (AOV) € 350,00; Commissione Media 8\% (€ 28,00 netti a vendita); Abbonamento B2C € 14,99.

\begin{table}[h]
\centering
\footnotesize
\renewcommand{\arraystretch}{1.4}
\begin{tabularx}{\textwidth}{|l|r|r|r|X|}
\hline
\textbf{METRICA} & \textbf{MESE 3} & \textbf{MESE 6} & \textbf{MESE 12} & \textbf{NOTE} \\
\hline
\textbf{Utenti Attivi (MAU)} & 100 & 500 & 2.500 & Crescita organica. \\
\hline
\textbf{Abbonati B2C (5\%)} & 5 & 25 & 125 & Tasso conv. standard. \\
\hline
\textbf{Vendite B2B (2\%)} & 2 & 10 & 50 & Acquisti tracciati. \\
\hline
\hline
\textbf{Ricavi B2C (€15)} & € 75 & € 375 & € 1.875 & Ricorrenti (MRR). \\
\hline
\textbf{Ricavi B2B (€28)} & € 56 & € 280 & € 1.400 & Variabili. \\
\hline
\textbf{TOTALE RICAVI} & \textbf{€ 131} & \textbf{€ 655} & \textbf{€ 3.275} & Fatturato mensile. \\
\hline
\hline
\textit{Costi Fissi} & (€ 151) & (€ 151) & (€ 151) & Infrastruttura. \\
\hline
\textit{Costi AI (Var)} & (€ 20) & (€ 100) & (€ 500) & €0.20/utente. \\
\hline
\textbf{PROFITTO (Netto)} & \textbf{(€ 40)} & \textbf{€ 404} & \textbf{€ 2.624} & Margine operativo. \\
\hline
\end{tabularx}
\caption{Proiezione P\&L Mensile (Scenario Realistico)}
\end{table}

\subsection{Analisi del Break-Even}
Con l'attuale struttura a costi fissi minimi (grazie all'assenza di stipendi dev), il progetto raggiunge il \textbf{Break-Even Point (Pareggio)} già intorno al \textbf{Mese 4-5}, con soli \textbf{150 utenti attivi}.

\section{Rischi e Mitigazione}

\begin{itemize}
    \item \textbf{Rischio:} \textit{Copia} da parte di Google.
    \item \textbf{Mitigazione:} Google non può replicare facilmente la \textit{Cura Umana} su scala globale senza distruggere il suo modello di business algoritmico neutro. La nostra difesa è la soggettività autoriale del dato.
    \item \textbf{Rischio:} Obsolescenza dei dati (negozi che chiudono).
    \item \textbf{Mitigazione:} Sistema di \textit{Feedback Loop} dagli utenti (segnalazioni) + integrazione automatica con Google Places API per rilevare lo status \textit{Permanently Closed}.
\end{itemize}
\chapter{Roadmap Operativa: From Zero to God Mode}

\section{Strategia di Esecuzione: The 12-Month Sprint}

La realizzazione di \textbf{Project A.R.B.O.R.} segue una metodologia Agile modificata per il Deep Tech. Non rilasceremo un prodotto incompleto, ma costruiremo strati di solidità progressiva.

La roadmap è divisa in 4 fasi trimestrali (Q1-Q4).

\section{Q1: The Foundation (Mesi 1-3)}
Obiettivo: Costruire l'infrastruttura dati e la pipeline di ingestione. Alla fine di questa fase, il sistema deve \textit{sapere} le cose, anche se non sa ancora \textit{parlare}.

\subsection{Mese 1: Infrastructure Setup}
\begin{itemize}
    \item \textbf{Cloud Environment:} Setup dei container Docker su Railway/AWS.
    \item \textbf{Database Trinity:} Inizializzazione delle istanze:
    \begin{itemize}
        \item \textbf{PostgreSQL:} Migrazioni schema SQL (Venues, Locations).
        \item \textbf{Qdrant:} Configurazione cluster Rust e indici HNSW.
        \item \textbf{Neo4j:} Definizione nodi e relazioni del Grafo.
    \end{itemize}
    \item \textbf{Repo Setup:} Configurazione CI/CD (GitHub Actions) e ambiente Python/Poetry.
\end{itemize}

\subsection{Mese 2: The Ingestion Engine (ETL)}
\begin{itemize}
    \item \textbf{Scraper Development:} Sviluppo dei bot Python per estrarre dati da Google Maps e Web.
    \item \textbf{Vision AI Integration:} Implementazione di GPT-4o Vision per analizzare le foto dei negozi e generare i primi \textit{Vibe Scores} automatici.
    \item \textbf{Curator Dashboard (v0.1):} Rilascio interno del pannello di controllo (Retool) per permettere al team editoriale di validare i dati scaricati.
\end{itemize}

\subsection{Mese 3: Data Population (Milano Pilot)}
\begin{itemize}
    \item \textbf{Massive Ingestion:} Caricamento dei primi 500 negozi su Milano (Focus: Sartoria, Artigianato, Hospitality).
    \item \textbf{Embedding Tuning:} Test dei vettori su Qdrant per assicurarsi che \textit{Elegante} e \textit{Formale} siano matematicamente vicini.
    \item \textbf{Graph Linking:} Creazione manuale/assistita delle prime relazioni storiche su Neo4j (e.g. \textit{Maestro X ha formato Sarto Y}).
\end{itemize}

\section{Q2: The Brain \& Logic (Mesi 4-6)}
Obiettivo: Sviluppare l'intelligenza. Alla fine di questa fase, il sistema sa ragionare e rispondere via API.

\subsection{Mese 4: The Agentic Swarm}
\begin{itemize}
    \item \textbf{LangGraph Setup:} Implementazione dell'orchestratore a grafo.
    \item \textbf{Agent Development:}
    \begin{itemize}
        \item Vector Agent: Connessione a Qdrant.
        \item Metadata Agent: Connessione a Postgres (SQL Tools).
        \item Historian Agent: Connessione a Neo4j (Cypher Tools).
    \end{itemize}
\end{itemize}

\subsection{Mese 5: The Curator Persona}
\begin{itemize}
    \item \textbf{System Prompt Engineering:} Raffinamento del tono di voce \textit{Porfirio}.
    \item \textbf{Intent Router:} Addestramento del classificatore per capire le intenzioni dell'utente.
    \item \textbf{Redis Caching:} Implementazione della Semantic Cache per ridurre i costi API e latenza.
\end{itemize}

\subsection{Mese 6: Internal Alpha (Dogfooding)}
\begin{itemize}
    \item \textbf{API Release (v1.0):} Rilascio degli endpoint FastAPI stabili.
    \item \textbf{Stress Test:} Simulazione di 100 utenti concorrenti per testare la tenuta di Qdrant e Postgre.g.
    \item \textbf{Quality Audit:} Il team di Porfirio Magazine testa le risposte per verificare l'accuratezza stilistica.
\end{itemize}

\section{Q3: The Experience (Mesi 7-9)}
Obiettivo: Dare un volto al sistema. Sviluppo delle interfacce utente Web e Mobile.

\subsection{Mese 7: Web App Integration}
\begin{itemize}
    \item \textbf{Next.js Development:} Sviluppo dell'interfaccia Chat e integrazione nel sito esistente di Porfirio Magazine.
    \item \textbf{Mapbox Integration:} Visualizzazione dei risultati su mappa interattiva customizzata (colori scuri/brandizzati).
    \item \textbf{Auth Integration:} Collegamento con Supabase Auth per gestire gli accessi.
\end{itemize}

\subsection{Mese 8: Mobile App (Flutter)}
\begin{itemize}
    \item \textbf{Core Development:} Porting delle funzionalità chat su iOS/Android.
    \item \textbf{Geolocation Features:} Implementazione della funzione \textit{Near Me} e notifiche push geolocalizzate.
    \item \textbf{App Store Submission:} Preparazione burocratica per Apple e Google Store.
\end{itemize}

\subsection{Mese 9: Closed Beta (\textit{The Vetted Club})}
\begin{itemize}
    \item \textbf{Soft Launch:} Invito a 500 utenti selezionati (HNWI / Amici del Brand).
    \item \textbf{Feedback Loop:} Raccolta bug e ottimizzazione UX.
    \item \textbf{Re-Ranking Tuning:} Ottimizzazione dell'algoritmo Cohere basata sui clic reali degli utenti.
\end{itemize}

\section{Q4: Launch \& Scale (Mesi 10-12)}
Obiettivo: Apertura al mercato, monetizzazione e scalabilità.

\subsection{Mese 10: Public Launch}
\begin{itemize}
    \item \textbf{Marketing Campaign:} Lancio ufficiale su Porfirio Magazine e canali social.
    \item \textbf{Paywall Activation:} Attivazione delle feature Premium (Concierge Mode).
    \item \textbf{City Expansion:} Apertura dei dati per Roma e Londra.
\end{itemize}

\subsection{Mese 11: B2B API Pilot}
\begin{itemize}
    \item \textbf{Documentation:} Rilascio della documentazione API per sviluppatori terzi.
    \item \textbf{Pilot Partners:} Integrazione del motore A.R.B.O.R. nel sito di un partner selezionato (e.g. Catena Hotel Lusso).
\end{itemize}

\subsection{Mese 12: Optimization & Rust Migration}
\begin{itemize}
    \item \textbf{Performance Review:} Analisi dei colli di bottiglia.
    \item \textbf{Rust Rewrite:} Inizio della migrazione dei microservizi critici da Python a Rust per ridurre i costi cloud del 50\%.
\end{itemize}

\section{Milestones Tecniche (KPIs)}

\begin{table}[h]
\centering
\begin{tabularx}{\textwidth}{|l|X|l|}
\hline
\textbf{Fase} & \textbf{Deliverable} & \textbf{Success Metric} \\
\hline
\textbf{Q1} & Database popolato (Milano) & 500+ Negozi Vetted \\
\hline
\textbf{Q2} & API Funzionante & Risposta < 2.5 sec \\
\hline
\textbf{Q3} & App Beta & Crash Rate < 0.1\% \\
\hline
\textbf{Q4} & Public Launch & 10k Monthly Active Users \\
\hline
\end{tabularx}
\caption{Key Performance Indicators per il primo anno}
\end{table}
\chapter{I Dati}
\section{Struttura Dati: Foglio \textit{Venues}}

Di seguito è riportata la definizione puntuale delle colonne presenti nel file master Excel. I campi sono divisi per competenza: \textbf{[MANUALE]} (compilati dal Curatore) e \textbf{[AI]} (generati automaticamente dagli script di arricchimento).

\subsection{Elenco Campi e Definizioni}

\begin{description}
    \item[\hl{Nome}] \texttt{[MANUALE]} \\
    Il nome ufficiale dell'attività commerciale o dell'artigiano. \\
    \textit{Facsimile: E. Marinella}

    \item[\hl{Città}] \texttt{[MANUALE]} \\
    La città di riferimento (senza indirizzo o CAP). \\
    \textit{Facsimile: Napoli}

    \item[\hl{Categoria}] \texttt{[MANUALE]} \\
    La categoria merceologica principale. I valori ammessi sono tassativi (in Inglese):
    \begin{itemize}
        \item \texttt{Accessories} (Accessori)
        \item \texttt{Books \& Music} (Libri e Musica)
        \item \texttt{Clothing} (Abbigliamento)
        \item \texttt{Food \& Drink} (Cibo e Bevande)
        \item \texttt{Footwear} (Calzature)
        \item \texttt{Fragrance \& Grooming} (Profumeria e Barba)
        \item \texttt{Motors} (Motori)
        \item \texttt{Tailoring} (Sartoria su misura)
    \end{itemize}

    \item[\hl{Note}] \texttt{[MANUALE]} \\
    Osservazioni libere del Curatore. Dettagli su esperienza, atmosfera, avvertenze. \\
    \textit{Facsimile: Negozio storico minuscolo, andare presto la mattina per evitare la folla.}

    \item[\hl{Genere}] \texttt{[MANUALE]} \\
    Il target di genere del negozio. Codici ammessi:
    \begin{itemize}
        \item \texttt{M}: Uomo (Men)
        \item \texttt{W}: Donna (Women)
        \item \texttt{MW}: Uomo e Donna
        \item \texttt{NC}: Non Classificabile / Neutro (es. Cibo, Motori)
    \end{itemize}

    \item[\hl{Stile}] \texttt{[MANUALE]} \\
    Il livello di formalità. Codici ammessi:
    \begin{itemize}
        \item \texttt{Casual}
        \item \texttt{Formal}
        \item \texttt{Casual and Formal}
        \item \texttt{Neutral} (per categorie non vestiarie)
    \end{itemize}

    \item[\hl{Link Maps}] \texttt{[MANUALE]} \\
    URL diretto alla scheda Google Maps. \\
    \textit{Facsimile: https://maps.app.goo.gl/...}

    \item[\hl{Referente}] \texttt{[MANUALE]} \\
    Nome del proprietario, store manager o commesso di fiducia da cercare. \\
    \textit{Facsimile: Maurizio}

    \item[\hl{Verificato}] \texttt{[MANUALE]} \\
    Indica se il luogo è stato visitato fisicamente dal Curatore.
    \begin{itemize}
        \item \texttt{SI}: Visitato e approvato.
        \item \texttt{No}: In lista desideri o segnalato da terzi (da verificare).
    \end{itemize}

    \item[\hl{Voto}] \texttt{[MANUALE]} \\
    Giudizio sintetico del Curatore (vedi Legenda Voto).

    \item[\hl{Founder Tags}] \texttt{[MANUALE]} \\
    Tag \textit{di pancia} che descrivono l'esperienza umana e dettagli non trovabili online. \\
    \textit{Facsimile: Istituzionale, Caotico, Top Quality, Scorbutico}

    \item[\hl{Prezzo}] \texttt{[AI]} \\
    Fascia di prezzo stimata (vedi Legenda Prezzo).

    \item[\hl{Vibe Tags}] \texttt{[AI]} \\
    Lista di 15-20 aggettivi in Inglese generati dall'AI per descrivere atmosfera e materiali. \\
    \textit{Facsimile: Heritage, Silk, Neapolitan, Bespoke, Wood-paneled}

    \item[\hl{Signature Items}] \texttt{[AI]} \\
    I 3-5 prodotti o servizi iconici per cui il luogo è famoso. \\
    \textit{Facsimile: Cravatta 7 Pieghe, Sciarpe Seta}

    \item[\hl{Target}] \texttt{[AI]} \\
    Il profilo cliente ideale (vedi Legenda Target).

    \item[\hl{Visual Style}] \texttt{[AI]} \\
    Breve descrizione dell'estetica e dell'arredamento. \\
    \textit{Facsimile: Boutique storica in legno scuro con vetrine d'epoca.}

    \item[\hl{Storia}] \texttt{[AI]} \\
    Snippet storico sulla fondazione o reputazione. \\
    \textit{Facsimile: Fondata nel 1914, fornitore ufficiale Real Casa inglese.}
\end{description}

\subsection{Legenda dei Valori Chiave}

Di seguito la spiegazione dettagliata delle scale di valutazione utilizzate.

\subsubsection{Legenda Voto (1-5)}
Il voto è espresso su una scala relativa interna alla selezione A.R.B.O.R.
Poiché il database è un \textit{Walled Garden}, la sola presenza in lista implica già il superamento della soglia di qualità. Pertanto, il voto \textbf{1} non indica insufficienza, ma il livello base di ingresso nella nostra selezione.

\begin{itemize}
    \item \textbf{1 (Valido):} Ha meritato l'ingresso in lista. Un indirizzo corretto, onesto e affidabile nel suo genere, pur senza eccellere in unicità.
    \item \textbf{2 (Buono):} Qualità superiore alla media. Un posto dove si torna volentieri, con un prodotto solido e ben eseguito.
    \item \textbf{3 (Ottimo):} Il \textit{Gold Standard}. Eccellenza tecnica e stilistica. È il punto di riferimento per la sua categoria.
    \item \textbf{4 (Eccellente):} Esperienza memorabile. Oltre al prodotto perfetto, c'è un fattore \textit{X} (storia, atmosfera, servizio) che lo rende speciale.
    \item \textbf{5 (Capolavoro / Icona):} L'apice assoluto. Un'istituzione sacra, un luogo di pellegrinaggio per gli intenditori. Perfezione indiscutibile.
\end{itemize}

\subsubsection{Legenda Prezzo (1-5)}
Indica il posizionamento economico rispetto alla media di mercato della categoria.
\begin{itemize}
    \item \textbf{1 (Economico):} Prezzi bassi, affari, street food economico.
    \item \textbf{2 (Accessibile):} Prezzi medi, buon rapporto qualità/prezzo.
    \item \textbf{3 (Premium):} Fascia alta ma standard. Costoso ma giustificato.
    \item \textbf{4 (Lusso):} Prezzi elevati, brand prestigiosi, materiali pregiati.
    \item \textbf{5 (Ultra-Lusso / Bespoke):} Prezzi senza limite. Alta sartoria, gioielleria, servizi esclusivi su misura.
\end{itemize}

\subsubsection{Legenda Target (Profilazione AI)}
Indica a quale tipologia di utente è più adatto il luogo, per guidare il Router dell'AI.
\begin{itemize}
    \item \textbf{Expert Only:} Luoghi intimidatori, nascosti o tecnici. Spesso su appuntamento, senza vetrina o con barriere linguistiche/culturali. Richiedono competenza per essere apprezzati (e per non essere trattati male).
    \item \textbf{Enthusiast:} Luoghi di alta qualità e passione. Accoglienti ma specifici. Richiedono un interesse per la materia, ma il personale è disposto a spiegare.
    \item \textbf{Tourist Friendly:} Luoghi facili, centrali, con personale che parla inglese e servizi Tax Free. Esperienza d'acquisto fluida e senza stress.
    \item \textbf{Local Gem:} Istituzioni di quartiere. Non necessariamente lusso, ma autentici e frequentati dai residenti. Il posto \textit{vero}.
    \item \textbf{High Spender:} Luoghi focalizzati sul VIP treatment. Lusso sfrenato, servizio impeccabile, ambiente esclusivo. Adatto a chi vuole spendere per essere coccolato.
\end{itemize}

\section{Struttura Dati: Foglio Brands}

Questa tabella gestisce le entità astratte (i Marchi). A differenza dei Negozi (luoghi fisici), i Brand rappresentano concetti, storia e prodotti che possono essere venduti in più luoghi.

\subsection{Elenco Campi e Definizioni}

\begin{description}
    \item[\hl{Nome}] \texttt{[MANUALE]} \\
    Il nome ufficiale del Brand o dell'Azienda produttrice. \\
    \textit{Facsimile: Barbour}

    \item[\hl{Categoria}] \texttt{[MANUALE]} \\
    La categoria merceologica principale (stessa tassonomia di Venues). \\
    \textit{Facsimile: Clothing}

    \item[\hl{Rivenditori}] \texttt{[MANUALE] - \textbf{CRITICO}} \\
    Elenco dei negozi fisici (presenti nel foglio Venues) che vendono questo brand. I nomi devono corrispondere esattamente. Separare con virgola. \\
    \textit{Facsimile: Davide Cenci, WP Store, Barbour Store Roma}

    \item[\hl{Note}] \texttt{[MANUALE]} \\
    Opinione del Curatore sul brand (evoluzione della qualità, reputazione attuale). \\
    \textit{Facsimile: Qualità del cotone cerato calata negli anni, ma resta un'icona indistruttibile.}

    \item[\hl{Link Sito}] \texttt{[MANUALE]} \\
    Sito web ufficiale del brand. \\
    \textit{Facsimile: www.barbour.com}

    \item[\hl{Founder Tags}] \texttt{[MANUALE]} \\
    Tag di pancia sulla percezione del brand. \\
    \textit{Facsimile: Indistruttibile, British, Inflazionato, Outdoor}

    \item[\hl{Nazione}] \texttt{[AI]} \\
    Paese di origine o sede principale del brand. \\
    \textit{Facsimile: Regno Unito}

    \item[\hl{Prezzo}] \texttt{[AI]} \\
    Posizionamento di prezzo medio del brand (1-5).

    \item[\hl{Vibe Tags}] \texttt{[AI]} \\
    Nuvola di aggettivi che descrivono l'estetica del brand. \\
    \textit{Facsimile: Countryside, Rainy, Waxed Cotton, Hunting, Royal Warrant}

    \item[\hl{Signature Items}] \texttt{[AI]} \\
    I prodotti più celebri prodotti dal brand. \\
    \textit{Facsimile: Bedale Jacket, Beaufort, Tartan Scarf}

    \item[\hl{Target}] \texttt{[AI]} \\
    A chi si rivolge il brand (stessa legenda di Venues). \\
    \textit{Facsimile: Enthusiast}

    \item[\hl{Stile}] \texttt{[AI]} \\
    Definizione sintetica dello stile. \\
    \textit{Facsimile: English Country}

    \item[\hl{Storia}] \texttt{[AI]} \\
    Snippet storico sulla fondazione. \\
    \textit{Facsimile: Fondata nel 1894 a South Shields, famosa per le giacche cerate.}
\end{description}

\section{Data Interconnection: La Genesi del Grafo}

Il vero valore di A.R.B.O.R. risiede nella capacità di collegare le due tabelle (\textit{Venues} e \textit{Brands}) per creare un **Knowledge Graph** navigabile.

\subsection{Il Ponte Logico: La Colonna Rivenditori}
La connessione avviene tramite la colonna \texttt{Rivenditori} nel foglio Brands.
Lo script di ingestione (\texttt{ingest\_master.py}) esegue la seguente logica di collegamento:

\begin{enumerate}
    \item Legge il nome del Brand (es. \textbf{Barbour}).
    \item Legge la lista dei Rivenditori (es. \textbf{Davide Cenci}).
    \item Cerca nel database dei Negozi se esiste un nodo chiamato Davide Cenci.
    \item Se esiste, crea una relazione direzionale nel grafo Neo4j.
\end{enumerate}

\subsection{Tipologie di Relazioni (Graph Edges)}
Il sistema è abbastanza intelligente da distinguere il tipo di relazione basandosi sui \textit{Founder Tags} del negozio.

\begin{itemize}
    \item \textbf{Relazione Standard (SELLS):}
    Se il negozio è un multimarca.
    \[ (:Venue \text{ Davide Cenci}) \xrightarrow{\text{SELLS\_BRAND}} (:Brand \text{ Barbour}) \]
    \textit{Significato:} Qui puoi comprare questo brand.

    \item \textbf{Relazione HQ (IS\_HQ\_OF):}
    Se il negozio ha nei Founder Tags parole come Flagship, Factory o Monobrand.
    \[ (:Venue \text{ Barbour Store Roma}) \xrightarrow{\text{IS\_HQ\_OF}} (:Brand \text{ Barbour}) \]
    \textit{Significato:} Questa è la casa madre o la rappresentanza ufficiale del brand.
\end{itemize}

\subsection{Vantaggio per l'Utente}
Questa struttura permette all'AI di rispondere a domande complesse di secondo livello:
\begin{quote}
    \textit{Mi piace lo stile di Barbour (Brand), ma sono a Roma. Portami in un negozio (Venue) che abbia quella stessa atmosfera (Vibe), anche se vende altri marchi.}
\end{quote}
Il sistema naviga dal Brand ai suoi Rivenditori, analizza il Vibe dei Rivenditori e trova luoghi simili vettorialmente.


% =========================
% Appendici
% =========================
\appendix

\chapter{Appendici Tecniche e Specifiche API}

\section{Appendix A: AI System Prompts (Proprietary Logic)}

In questa sezione divulghiamo la logica di istruzione (Prompt Engineering) utilizzata per guidare i modelli LLM. Questi prompt rappresentano parte della proprietà intellettuale (IP) del progetto.

\subsection{A.1 The Curator Persona (System Prompt)}
Questo è il prompt di base che definisce la personalità e i vincoli dell'assistente durante la chat con l'utente.

\begin{lstlisting}[frame=single, breaklines=true, basicstyle=\ttfamily\footnotesize]
ROLE: 
You are \textit{The Curator}, an elite personal shopper and lifestyle expert. You possess the combined knowledge of a bespoke tailor, an interior designer, and a local historian.

CONSTRAINTS:
1. TRUTH: You answer ONLY based on the context provided in the RAG retrieval. If the context is empty, admit you don't know. Do not hallucinate shops not in the database.
2. TONE: Sophisticated, concise, warm but professional. Avoid generic marketing fluff like \textit{stunning} or \textit{breathtaking}. Use technical vocabulary (e.g., \textit{Goodyear welted}, \textit{Unlined}, \textit{Full canvas}).
3. FORMAT: When recommending a place, always provide the \textit{Match Score} and the specific reason why it fits the user's vibe.

TASK:
The user is looking for a recommendation. Analyze the provided Context Data (JSON) and synthesize the best 3 options. Explain the trade-offs between them (e.g., \textit{Option A is more formal, while Option B is more fashion-forward}).
\end{lstlisting}

\subsection{A.2 The Vibe Extractor (Ingestion Prompt)}
Questo prompt viene utilizzato dall'agente Python in fase di analisi delle recensioni per popolare il database.

\begin{lstlisting}[frame=single, breaklines=true, basicstyle=\ttfamily\footnotesize]
TASK:
Analyze the following raw reviews and images description of a venue. Extract key dimensional scores (0-100) and semantic tags.

INPUT DATA:
[...Raw Reviews Text...]
[...Image Analysis Description...]

OUTPUT FORMAT (JSON ONLY):
{
  "dimensions": {
    "formality": <0-100>,      // 0=Streetwear, 100=Black Tie
    "craftsmanship": <0-100>,  // 0=Industrial, 100=Handmade on site
    "price\_value": <0-100>,    // 0=Overpriced, 100=Bargain
    "atmosphere": <0-100>      // 0=Chaotic/Loud, 100=Zen/Private
  },
  "tags": ["<tag1>", "<tag2>", "<tag3>"], // Max 5 tags, strictly from the Allowed Ontology
  "summary": "<One sentence expert summary>"
}
\end{lstlisting}

\section{Appendix B: API Data Structures}

\subsection{B.1 The Vibe DNA (JSONB Schema)}
Esempio reale di come viene strutturato il campo `vibe\_dna` nel database PostgreSQL. Questo è l'oggetto che permette il calcolo della similarità.

\begin{lstlisting}[language=json, frame=single, basicstyle=\ttfamily\footnotesize]
{
  "venue\_id": "550e8400-e29b-41d4-a716-446655440000",
  "name": "Sartoria Partenopea",
  "dimensions": {
    "formality": 85,
    "craftsmanship": 95,
    "trendiness": 15,
    "exclusivity": 70
  },
  "semantic\_anchors": [
    "Neapolitan Shoulder",
    "Bespoke Service",
    "Hidden Gem",
    "Old Money Aesthetic"
  ],
  "context\_rules": {
    "best\_for": ["Wedding", "Business Formal"],
    "avoid\_for": ["Casual Friday", "Last Minute Gift"]
  }
}
\end{lstlisting}

\section{Appendix C: Stack di Sicurezza e Privacy (GDPR)}

Trattando preferenze personali e dati di localizzazione, l'architettura segue i principi di Privacy by Design.

\begin{itemize}
    \item \textbf{Data Minimization:} L'LLM non riceve mai l'ID utente reale, ma un session\_token effimero.
    \item \textbf{Zero Retention:} I provider AI (OpenAI/Cohere) sono configurati con policy \textit{Zero Data Retention}, garantendo che le chat degli utenti non vengano usate per addestrare i loro modelli futuri.
    \item \textbf{Encryption:} Tutti i dati sono criptati in transito (TLS 1.3) e a riposo (AES-256 su DB Supabase).
\end{itemize}

% =========================
% Conclusioni
% =========================
\include{conclusioni}

\end{document}