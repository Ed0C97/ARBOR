% ============================================================================
% Chapter 25: Glossary
% ============================================================================

\chapter{Glossary}
\label{ch:glossary}

This glossary defines key terms used throughout the ARBOR documentation.


\section{Architecture Terms}

\begin{description}
    \item[Agentic Orchestration] The pattern of using autonomous AI agents coordinated by a supervisor to accomplish complex tasks, as implemented via LangGraph in ARBOR.
    
    \item[GraphRAG] Graph-based Retrieval-Augmented Generation; using knowledge graph traversal to provide context for LLM responses.
    
    \item[Hybrid Search] Combining dense vector similarity (semantic) and sparse vector matching (keyword) for retrieval, using RRF fusion.
    
    \item[Knowledge Trinity] ARBOR's three-database architecture: PostgreSQL for structured data, Qdrant for vector search, Neo4j for graph relationships.
    
    \item[LLM Gateway] Centralized layer managing LLM access, including routing, caching, guardrails, and observability.
    
    \item[Polyglot Persistence] Using multiple database technologies, each optimized for specific access patterns.
\end{description}


\section{Domain Concepts}

\begin{description}
    \item[Curator] Human expert who validates entities and maintains knowledge base quality; also the AI persona that synthesizes recommendations.
    
    \item[Domain] A configured vertical such as lifestyle, hospitality, or real estate, with its own categories, dimensions, and relationship types.
    
    \item[Entity] Any discoverable item in the knowledge base: a store, restaurant, professional, brand, or other domain-specific object.
    
    \item[Vibe DNA] Multi-dimensional scoring representing an entity's qualitative characteristics (e.g., formality, craftsmanship, price level).
    
    \item[Vibe Dimension] A single axis of the Vibe DNA, defined per domain (e.g., ``formality: 0-100'').
\end{description}


\section{Technical Components}

\begin{description}
    \item[Agent State] The TypedDict structure containing all information that flows through the agent graph during query processing.
    
    \item[Curator Agent] The agent responsible for synthesizing retrieval results into natural language recommendations.
    
    \item[Enrichment Orchestrator] Component that coordinates AI analysis (vision, review, embedding) during entity ingestion.
    
    \item[GenericEntityRepository] Schema-agnostic repository pattern enabling domain flexibility without code changes.
    
    \item[Historian Agent] Agent that queries Neo4j for relationship context around retrieved entities.
    
    \item[Intent Router] Agent that classifies user queries and extracts structured filters.
    
    \item[LangGraph] LangChain framework for building stateful, multi-agent workflows as graphs.
    
    \item[Langfuse] LLM observability platform for tracing, cost tracking, and evaluation.
    
    \item[LiteLLM] Multi-provider LLM gateway enabling unified access to OpenAI, Anthropic, Azure, etc.
    
    \item[Metadata Agent] Agent that queries PostgreSQL for entities matching structured filters.
    
    \item[NeMo Guardrails] NVIDIA framework for adding safety rails to LLM applications.
    
    \item[Qdrant] Vector database used for semantic similarity search.
    
    \item[Reciprocal Rank Fusion (RRF)] Algorithm for combining ranked lists from multiple retrieval sources.
    
    \item[Temporal.io] Durable workflow execution platform used for ingestion and background processing.
    
    \item[Vector Agent] Agent that performs semantic similarity search in Qdrant.
\end{description}


\section{ML Terms}

\begin{description}
    \item[A/B Testing] Controlled experimentation comparing variants to measure impact.
    
    \item[Causal Inference] Statistical methods for estimating treatment effects.
    
    \item[Drift Detection] Monitoring for changes in data distributions that may affect model performance.
    
    \item[Feature Store] Real-time serving infrastructure for ML features.
    
    \item[Knowledge Distillation] Compressing large models into smaller, efficient variants.
    
    \item[Reranking] Second-stage ranking refinement after initial retrieval.
    
    \item[RLHF] Reinforcement Learning from Human Feedback; using human preferences to improve model outputs.
    
    \item[SHAP] SHapley Additive exPlanations; method for explaining individual predictions.
\end{description}


\section{Infrastructure Terms}

\begin{description}
    \item[Circuit Breaker] Pattern that prevents cascading failures by stopping requests to failing services.
    
    \item[HPA] Horizontal Pod Autoscaler; Kubernetes mechanism for scaling based on metrics.
    
    \item[Istio] Service mesh providing mTLS, traffic management, and observability.
    
    \item[OpenTelemetry] Observability framework for distributed tracing and metrics.
    
    \item[PgBouncer] Connection pooler for PostgreSQL.
    
    \item[StatefulSet] Kubernetes workload for stateful applications like databases.
\end{description}


\section{Abbreviations}

\begin{table}[H]
\centering
\begin{tabular}{ll}
\toprule
\textbf{Abbreviation} & \textbf{Meaning} \\
\midrule
AAA & Arrange-Act-Assert \\
API & Application Programming Interface \\
CDC & Change Data Capture \\
CDN & Content Delivery Network \\
CI/CD & Continuous Integration/Continuous Deployment \\
E2E & End-to-End \\
GNN & Graph Neural Network \\
JWT & JSON Web Token \\
LLM & Large Language Model \\
OIDC & OpenID Connect \\
P95 & 95th Percentile \\
RAG & Retrieval-Augmented Generation \\
RBAC & Role-Based Access Control \\
RPS & Requests Per Second \\
SLI/SLO & Service Level Indicator/Objective \\
SSR & Server-Side Rendering \\
WAF & Web Application Firewall \\
\bottomrule
\end{tabular}
\caption{Common abbreviations}
\end{table}
