% ============================================================================
% Chapter 24: Configuration Reference
% ============================================================================

\chapter{Configuration Reference}
\label{ch:configuration}

This appendix documents ARBOR's configuration files and environment variables.


\section{Domain Configuration}

\subsection{Domain Profile Structure}

Domain profiles define domain-specific configuration:

\begin{lstlisting}[style=yamlstyle, caption={Domain profile example}]
domain:
  id: lifestyle
  name: "Lifestyle & Fashion"
  description: "Curated discovery for lifestyle entities"

categories:
  - id: tailoring
    name: "Tailoring"
    subcategories:
      - id: bespoke
        name: "Bespoke"
      - id: mtm
        name: "Made-to-Measure"
  - id: shoes
    name: "Footwear"

dimensions:
  formality:
    name: "Formality"
    description: "0 = Casual, 100 = Formal"
    weight: 1.0
  craftsmanship:
    name: "Craftsmanship"
    description: "0 = Mass-produced, 100 = Artisanal"
    weight: 1.2

styles:
  - id: neapolitan
    name: "Neapolitan"
    keywords: ["spalla scesa", "unconstructed"]
  - id: english
    name: "English/Savile Row"
    keywords: ["structured", "padded shoulder"]

relationship_types:
  - sells_brand
  - trained_by
  - has_style

curator_persona:
  name: "The Style Curator"
  voice: "Sophisticated but approachable"
\end{lstlisting}


\section{Source Schema Configuration}

\subsection{Data Source Schema}

SOURCE\_SCHEMA\_CONFIG defines data source mappings:

\begin{lstlisting}[style=yamlstyle, caption={Source schema configuration}]
source_schema:
  name: "google_maps"
  entity_mapping:
    name: "$.name"
    external_id: "$.place_id"
    location:
      lat: "$.geometry.location.lat"
      lng: "$.geometry.location.lng"
    attributes:
      rating: "$.rating"
      review_count: "$.user_ratings_total"
      price_level: "$.price_level"
  
  category_mapping:
    "clothing_store": "fashion"
    "restaurant": "dining"
    "cafe": "coffee"
\end{lstlisting}


\section{Environment Variables}

\subsection{Core Settings}

\begin{table}[H]
\centering
\begin{tabularx}{\textwidth}{lL{7cm}}
\toprule
\textbf{Variable} & \textbf{Description} \\
\midrule
\code{ENVIRONMENT} & Environment name (dev/staging/prod) \\
\code{DEBUG} & Enable debug mode \\
\code{LOG\_LEVEL} & Logging level \\
\code{SECRET\_KEY} & Application secret key \\
\bottomrule
\end{tabularx}
\caption{Core environment variables}
\end{table}

\subsection{Database Settings}

\begin{table}[H]
\centering
\begin{tabularx}{\textwidth}{lL{7cm}}
\toprule
\textbf{Variable} & \textbf{Description} \\
\midrule
\code{DATABASE\_URL} & PostgreSQL connection string \\
\code{QDRANT\_URL} & Qdrant server URL \\
\code{QDRANT\_API\_KEY} & Qdrant API key \\
\code{NEO4J\_URI} & Neo4j connection URI \\
\code{NEO4J\_USER} & Neo4j username \\
\code{NEO4J\_PASSWORD} & Neo4j password \\
\code{REDIS\_URL} & Redis connection string \\
\bottomrule
\end{tabularx}
\caption{Database environment variables}
\end{table}

\subsection{LLM Settings}

\begin{table}[H]
\centering
\begin{tabularx}{\textwidth}{lL{7cm}}
\toprule
\textbf{Variable} & \textbf{Description} \\
\midrule
\code{OPENAI\_API\_KEY} & OpenAI API key \\
\code{ANTHROPIC\_API\_KEY} & Anthropic API key \\
\code{AZURE\_API\_KEY} & Azure OpenAI key \\
\code{AZURE\_API\_BASE} & Azure endpoint URL \\
\code{DEFAULT\_LLM\_MODEL} & Default model for inference \\
\bottomrule
\end{tabularx}
\caption{LLM environment variables}
\end{table}

\subsection{Authentication}

\begin{table}[H]
\centering
\begin{tabularx}{\textwidth}{lL{7cm}}
\toprule
\textbf{Variable} & \textbf{Description} \\
\midrule
\code{AUTH0\_DOMAIN} & Auth0 tenant domain \\
\code{AUTH0\_AUDIENCE} & Auth0 API audience \\
\code{AUTH0\_CLIENT\_ID} & Auth0 client ID \\
\code{AUTH0\_CLIENT\_SECRET} & Auth0 client secret \\
\bottomrule
\end{tabularx}
\caption{Authentication environment variables}
\end{table}


\section{Prompt Templates}

\subsection{Template Directory}

Prompts are stored in \code{config/prompts/}:

\begin{lstlisting}[caption={Prompt directory structure}]
config/prompts/
|-- intent_classification.txt
|-- curator_synthesis.txt
|-- vibe_extraction.txt
|-- vision_analysis.txt
`-- entity_comparison.txt
\end{lstlisting}

\subsection{Template Variables}

Common template variables:

\begin{itemize}
    \item \code{\{domain\_name\}}: Current domain name
    \item \code{\{dimensions\}}: List of vibe dimensions
    \item \code{\{categories\}}: Available categories
    \item \code{\{query\}}: User query
    \item \code{\{context\}}: Retrieved context
    \item \code{\{persona\}}: Curator persona
\end{itemize}


\section{Feature Flags}

Feature flags control runtime behavior:

\begin{lstlisting}[style=yamlstyle, caption={Feature flags}]
features:
  semantic_cache: true
  graph_rag: true
  personalization: true
  experimental_reranker: false
  new_curator_prompt: false
\end{lstlisting}
